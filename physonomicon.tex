\documentclass{article}
\usepackage{tikz}
\usepackage{pgfplots}
\usepackage{wrapfig}
\usepackage[export]{adjustbox}[2011/08/13]
\usepackage{amsmath}
\usepackage{amssymb}
\usepackage{hyperref}
\usepackage{textcomp, gensymb}
\usepackage{graphicx}
\graphicspath{ { ./ }}
\usepackage{geometry}
 \geometry{
 a4paper,
 includeheadfoot,
 }
\pgfplotsset{width=10cm,compat=1.9}

\title{Physonomicon}
\author{Brasides}

\setlength{\jot}{10pt}
\numberwithin{equation}{section}

\begin{document}
\maketitle
\newpage
\tableofcontents

 \section{Derivations for One Dimensional Motion}
 \subsection{Average velocity $\bar{v}$ vs Instantaneous velocity $v$}
 \subsubsection{tl;dr}
 
\[
  \begin{aligned}
    &\bar{v} = \frac{x_2 - x_1}{t_2 - t_1} & avg\ velocity\\ 
    &x(t) = At^n & position\ as\ function\ of\ time \\ 
    & \bar{v} = \frac{x(t_2)-x(t_1)}{t_2 - t_1} & \\
    & & \\
    \hline{}\\
    & Definition\ of\ v(t):
    & v(t) = \lim_{\Delta t\to0} 
    \frac{x(t+\Delta t)-x(t)}{\Delta t}  
    \iff v(t) = \frac{dx}{dt}  \\
  \end{aligned}
\]
Where $A$ is a constant and $n$ is an integer.
 \subsubsection{Defining $\bar{v}$}
 Average velocity $\bar{v}$ is defined for position $x_1$ at time
 $t_1$ and position $x_2$ at time $t_2$:
 \[
   \bar{v} = \frac{x_2 - x_1}{t_2 - t_1}
\]
Because we may consider the position $x$ to be a function of time, we
may instead choose to rewrite this as:
 \[
    \bar{v} = \frac{x(t_2)-x(t_1)}{t_2 - t_1}
\]
Where $x(t) = At^{n} $
   
In order to go from $\bar{v}$ to instantaneous velocity $v$, we swap
in values that are more appropriate for calculus:\par 
\textit{Let }$t_1 = t$ \textit{ and } $t_2 = t+ \Delta t$\\
Now we may consider the limit as $\Delta t \to 0$, which we shall
define as instantaneous velocity $v(t)$:
 
\begin{equation}\label{eq:defnv}
    v(t) = \lim_{\Delta t\to0} 
    \frac{x(t+\Delta t)-x(t)}{\Delta t}  
    \iff v(t) = \frac{dx}{dt}
    = \frac{d}{dt}x(t)
  \end{equation}
Where the function $x(t)$ has the form:
Where $A$ is a constant and $n$ is an integer.

 \subsection{Average acceleration $\bar{a}$ vs Instantaneous
   acceleration $a$}
   \subsubsection{tl;dr}
\[
  \begin{aligned}
    &\bar{a} = \frac{v_2 - v_1}{t_2 - t_1} & avg\ velocity\\ 
    &v(t) = \frac{dx}{dt} & inst.\ velocity\ as\ function\ of\ time \\ 
    & \bar{a} = \frac{v(t_2)-v(t_1)}{t_2 - t_1} & \\
    & & \\
    \hline{}\\
    & Definition\ of\ a(t):
    & a(t) = \lim_{\Delta t\to0} 
    \frac{v(t+\Delta t)-v(t)}{\Delta t}  
    \iff a(t) = \frac{d}{dt}v(t) = \frac{d^2x}{dt^2}  \\
  \end{aligned}
\]
   \subsubsection{Defining average velocity $\bar{a}$}
   The definition of acceleration has a similar form to the definition
   of velocity above, and is written in its most simple terms as:\\ 
   \[
     \bar{a} = \frac{v_2 - v_1}{t_2 - t_1}
  \]
   Because we may consider instantaneous velocity as a function of
   time $v(t)$, we may instead choose to write this as:\\ 
   \[
     \bar{a} = \frac{v(t_2) - v(t_1)}{t_2 - t_1}
  \]
  Now we define the terms differently to more clearly apply the limit
  to arrive at the derivative:\\
  
 \[
   \begin{aligned}
     t_1 &= t \text{, } & t_2= t + \Delta t  \\
   \end{aligned}
 \]
 Now we may define the instantaneous acceleration $a$ to be the limit
 of the instantaneous velocity when the $\Delta t \to 0$:
 
 \begin{equation}\label{eq:defnacc}
    a = \lim_{x\to0}\frac{v(t + \Delta t) - v(t) }{\Delta t} 
    \iff a = \frac{d}{dt}v(t) \\
  \end{equation}
\newpage
 \subsection{Notation}
 As a simplification, we always take the initial time $t_{0}=0$ and define our
 equations with the supposition that acceleration is constant. Thus we write:
 
\[
  \begin{aligned}
    t &= t_f \\
    t &= t + 0 = t - t_{0} = \Delta t
  \end{aligned}
\]
 \begin{equation}
   \boxed{    \Delta t = t \label{time}}
 \end{equation}
\[
  \begin{aligned}
    \Delta x &= \Delta x - x_{0} \\
    \Delta v &= v - v_{0} \\
    \bar{a} &= a = constant
  \end{aligned}
\]
 \subsection{Deriving the first kinematic equation}
 
 \subsubsection{Position $x$ as a function of $\bar{v}$}
\[
  \begin{aligned}
    \bar{v} &= \frac{\Delta x}{\Delta t} \\
    \bar{v} &= \frac{x-x_{0}}{t} \\ 
    x &= x_{0}+\bar{v}t
  \end{aligned}
\]
\subsubsection{Final Velocity $v$ as a function of time $t$ and acceleration
$a=\bar{a}$}
 
\[
  \begin{aligned}
    a &= \frac{\Delta v}{\Delta t} & t = \Delta t \\
    a  &= \frac{v-v_{0}}{t}
  \end{aligned}
\]
 \begin{equation}
   v = v_{0} + at
 \end{equation}

 \subsubsection{Final Velocity $v$ as a function of distance $x$ and
 acceleration $a = \bar{a}$}
 Starting with
 \[
     v = v_{0}+at \\
\]
Rearranging for $t$
\[
  \begin{aligned}
    t &= \frac{v-v_{0}}{a}\\ 
  \end{aligned}
\]
By definition $\bar{v}$ 
\[
    \bar{v} = \frac{v_{0}+v}{2} 
\]
Now substitute into equation for position
 
\[
  \begin{aligned}
    x &= x_{0} + \bar{v}t \\
    x &= x_{0} + \frac{v_{0}+v}{2}*\frac{v - v_{0}}{a} \\
    x &= x_{0} + \frac{v^2-v_{0}^2}{2a}
  \end{aligned}
\]
Solving for $v^2$
 \[
    v^2 = v_{0}^2 + 2a(x-x_{0})
\]
Or
 \[
   \boxed{    v^2 = v_{0}^2 + 2a \Delta x}
\]
 \subsubsection{Deriving position $x$ as a function of velocity $v$, time $t$,
 and acceleration $a = \bar{a}$}
 We begin with position as a function of $\bar{v}$:
 \[
   x = x_{0} + \bar{v}t
\]
We know we want $v$, $t$, and $a$, so we start with the equation for $v$ and
rearrange to substitute $\bar{v}$:
 
\[
  \begin{aligned}
    v &= v_{0} + at \\
      & add\ v_{0}\ to\ both\ sides \\
    v+v_{0} &= 2v_{0} + at \\
    \frac{v+v_{0}}{2} &= v_{0} + \frac{at}{2} \\
                      & substitute\ \bar{v} = \frac{v+v_{0}}{2}\\ 
    \bar{v} &= v_{0} + \frac{1}{2}at \\
            & substite\\
            x &= x_{0} + (v_{0}+\frac{1}{2}at)t\\ 
  \end{aligned}
\]
 \[
   \boxed{                x = x_{0} + v_{0}t + \frac{1}{2}at^2}
\]
 \subsection{Kinematic Equations from Integral Calculus}
 \subsubsection{$v(t) = v_{0}+at$ by Integration}
 We defined the relationship between acceleration and velocity in
 (\ref{eq:defnacc}) as:
 \[
     a = \frac{d}{dt}v(t)
\] 
Switching sides, as we are more interested in $v(t)$, and rewriting $a$ as a
function of time $a(t)$:
\[
  \begin{aligned}
    \frac{d}{dt}v(t) &= a(t) \\
    \int \frac{d}{dt}v(t) \,dt &= \int a(t) \,dt + C_{1} \\
  \end{aligned}
\]
\begin{equation}\label{eq:vintegral}
  v(t) = \int a(t) \,dt + C_{1} \\ 
\end{equation}
 
For constant acceleration in (\ref{eq:vintegral}):
 
\[
  \begin{aligned}
    v(t) &= \int a \,dt = at + C_{1}\\
    v(0) = v_{0} &= a(0) + C_{1}\\ 
    v_{0} &= C_{1}\\ 
  \end{aligned}
\]
\begin{equation}\label{eq:vtrule}
  \therefore \, \boxed{v(t) = v_{0}+at}
\end{equation}

 \subsubsection{$x(t)$ by Integration}
 We defined the function of position in (\ref{eq:defnv}) as:
\[
  \begin{aligned}
    \frac{d}{dt}x(t) &= v(t) \\
    \int \frac{d}{dt}x(t) \,dt &= \int v(t) \,dt \\
  \end{aligned}
\]
\begin{equation}\label{eq:xintegral}
  x(t) = \int v(t) \,dt + C_{2} \\ 
\end{equation}
Substite equation for $v(t)$ from (\ref{eq:vtrule}) in (\ref{eq:xintegral})
 
\[
  \begin{aligned}
    x(t) &= \int v_{0}+at \,dt \\
    x(t) &= v_{0}t + \frac{1}{2}at^2 + C_{2} \\
    x(0) = x_{0} &= v_{0}(0) + \frac{1}{2}a(0)^2 + C_{2}\\
    x_{0} &= C_{2} \\
  \end{aligned}
\]
\begin{equation}
  \therefore \, \boxed{    x(t) = x_{0}+v_{0}t + \frac{1}{2}at^2}
\end{equation}
Example on pg 136 of the textbook.

 \section{Projectile Motion Derivations}
 Motion of an object when:
 
\[
  \begin{aligned}
    a_{x} &= 0 \\
  \end{aligned}
\]
 \subsection{Defining vector quantities of motion for two dimensional vectors}
  Where $x(t)$ and $y(t)$ are defined for two dimensional position vector
  $\vec{r}(t)$ and similarly $v_{x}(t)$ and $v_{y}(t)$ are defined for velocity
  vector $\vec{v}(t)$, while $\vec{a}$ is similar but with $a_x=0$:
  
 \[
   \begin{aligned}
    \vec{r}(t) &= x(t)\hat{i} + y(t)\hat{j} \\ 
    \vec{v}(t) &= v_x(t)\hat{i} + v_y(t)\hat{j}\\
    \vec{a}(t) &= 0\hat{i} + a_y(t)\hat{j}
   \end{aligned}
 \]
 \subsection{Rewriting kinematic equations for x and y directions where $a_x=0$}
\[
  \begin{aligned}
    x(t) &= x_{0} + \bar{v}_xt \\
    v_{x}(t) &= v_{0,x} \\
    x(t) &= x_{0} + v_{0,x} t + \frac{1}{2}a_xt^2 \\ 
    v_{x}^2(t) &=  v_{0,x}^2t\\
              &\ \\
    y(t) &= y_{0} + \bar{v}_yt \\
    v_{y}(t) &= v_{0,y} + a_{y}t^2 \\
    y(t) &= y_{0} + v_{0,y} t + \frac{1}{2}a_yt^2 \\
    v_{y}^2(t) &= v_{0,y}^2 + 2a_{y}(y-y_{0}) 
  \end{aligned}
\]
 
 \subsection{Derivations of Projectile Motion Equations}
 \subsubsection{Maximum height $h$}
 Requires knowing two of three: $y$, $v_{0,y}^2$, $g$\\
 For $v_{y}=0$, $y_{0}=0$, $a_{y} = -g$:
\[
  \begin{aligned}
    v_{y}^2 &= v_{0,y}^2 + 2a_{y}(y - y_{0}) \\
    0 &= v_{0,y}^2 + 2(-g)(y - 0) \\
    0 &= v_{0,y}^2 - 2gy
  \end{aligned}
\]
\begin{equation}\label{eq:projh}
  \therefore \, \boxed{y = \frac{v_{0,y}^2}{2g}}
\end{equation}

 \subsubsection{Direction of velocity $\theta_v$}
 The direction of velocity does not need to be derived, as it relies on the
 triangle formed by $v_x$ and $v_y$:
 \begin{equation}\label{eq:launchangle}
    \theta = \tan^-1 \bigg( \frac{v_y}{v_x} \bigg)
 \end{equation}
 \subsubsection{Time of Flight $T_{tof}$}
 For $y=y_{0}$, $a_{y} = -g$:
 
\[
  \begin{aligned}
    y - y_{0} &= v_{0,y}t + \frac{1}{2}a_{y}t^2 \\
    0 &= v_{0,y}t + \frac{1}{2}(-g)t^2 \\
    0 &= v_{0,y}t - \frac{1}{2}gt^2 & v_{0,y} = \sin(\theta_{0})v_{0}\\
    0 &= \sin(\theta_0)v_{0}t -\frac{1}{2}gt^2 \\
    0 &= t \bigg( \sin(\theta_0)v_{0} -\frac{1}{2}gt \bigg) \\
    0 &= \sin(\theta_{0}) - \frac{1}{2}gt \\
    \frac{1}{2}gt &= \sin(\theta_{0}) \\
    t &= \frac{2\sin(\theta_{0})}{g} \\
  \end{aligned}
\]
\begin{equation}\label{eq:tof}
   \boxed{    T_{tof} = \frac{2\sin(\theta_{0})}{g} \\}
 \end{equation}
 \subsubsection{Trajectory $y$ as a function of $x$ without time $t$}
 Need three of four: $y$, $v_{0}$, $\theta_{0}$, $x$
 For $a = -g$
 \[
    x = v_{0,x}t
\] 
Solving for $t$:
 
\[
  \begin{aligned}
    t &= \frac{x}{v_{0,x}} \\
    t &= \frac{x}{\cos(\theta_{0})v_{0}} \\
  \end{aligned}
\]
$y$-position:
 
\[
  \begin{aligned}
    y &= v_{0,y}t + \frac{1}{2}at^2
    y &= v_{0}\sin(\theta_{0})t - \frac{1}{2}gt^2 \\
  \end{aligned}
\]
Substitute $t$:
 
\[
  \begin{aligned}
    y &= v_{0}\sin(\theta_{0}) \bigg( \frac{x}{\cos(\theta_{0})v_{0}} \bigg) -
    \frac{1}{2}g \bigg( \frac{x}{\cos(\theta_{0})v_{0}} \bigg)^2 \\
  \end{aligned}
\]
\begin{equation}\label{eq:trajectory}
  \boxed{    y = \tan(\theta_{0})x - x^2 \bigg( \frac{g}{2(v_{0} \cos\theta_{0} )^2} \bigg)}
\end{equation}
 \subsubsection{Range}
 Starting with trajectory equation (\ref{eq:trajectory}) where $y=0$:
 
\[
  \begin{aligned}
    0 &= \tan(\theta_{0})x - x^2 \bigg( \frac{g}{2(v_{0} \cos\theta_{0} )^2} \bigg) \\
    x^2 \bigg( \frac{g}{2(v_{0} \cos\theta_{0} )^2} \bigg) &=
    \tan(\theta_{0})x\\
    x  &= \tan(\theta_{0}) \bigg( \frac{2(v_{0}\cos(\theta_{0})^2)}{g} \bigg) \\
    x &= \frac{v_{0}^2 2\sin(\theta_{0})\cos(\theta_{0})}{g}\\ 
  \end{aligned}
\]
  \begin{equation}\label{eq:range}
    \boxed{    x = R = \frac{v_{0}^2 \sin(2\theta_{0})}{g} }
  \end{equation}

\end{document}
